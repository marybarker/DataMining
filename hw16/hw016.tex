\documentclass[11pt]{article}
\usepackage{fullpage}
\usepackage{setspace}
\usepackage{amsmath}
\usepackage{fancyvrb}
\usepackage{enumerate}
\usepackage{pgfplots}
\usepackage{graphicx}
\usepackage{float}
\usepackage{multirow}
\usepackage[format=hang,labelsep=quad]{caption}
\usepackage{subfig}
\usepackage{array}
\usepackage{multirow}

\renewcommand\thesubfigure{\roman{subfigure}}


\begin{document}
\noindent\large{Math 5365}\\
\large{Data Mining 1}\\
\large{Homework 16}\\
\large{Mary Barker}
\doublespace
\begin{enumerate}
\item 
Load the fgl data set in the MASS library. 

 Split the data into 70\% training and 30\% testing data. 
 Predict the glass types in the test data using the following
 three methods, and compare the classification accuracies of 
 these methods.


\begin{enumerate}
\item
Build an SVM and use the predict command.

Using the basic SVM command with no altered parameters, the accuracy is 73.4375\%

\item 
 Write a script implementing the one-against-rest approach 
 with SVM's (much of the code you need is in the 'programming
 one-against-rest' section of chapter5.R)

The confusion matrix for this case is shown below. 

\begin{center}
\begin{tabular}{c c c c c c c}
&   &       &     & predy &      & \\
&   &       & Con & Head  & WinF & WinNF \\
&   & WinF  &  0  &  0    &  21  &    5 \\
&   & WinNF &  0  &  0    &   3  &   18 \\
& y & Veh   &  0  &  0    &   3  &    0 \\
&   & Con   &  2  &  0    &   0  &    1 \\
&   & Tabl  &  0  &  0    &   1  &    0 \\
&   & Head  &  0  &  9    &   1  &    0
\end{tabular}
\end{center}


\item
 Write a script implementing the one-against-one approach 
with SVM's.

The confusion matrix for this test is shown below.

\begin{center}
\begin{tabular}{c c c c c c c c}
 & & & & y & & & \\
     &      & WinF  & WinNF & Veh & Con & Tabl & Head \\
      & WinF  & 16 & 10 & 0 & 0 & 0 & 0 \\
      & WinNF & 2  & 19 & 0 & 0 & 0 & 0 \\
predy & Veh   & 2  & 1  & 0 & 0 & 0 & 0 \\
      & Con   & 0  & 1  & 0 & 2 & 0 & 0 \\
      & Tabl  & 0  & 0  & 0 & 0 & 1 & 0 \\
      & Head  & 1  & 0  & 0 & 0 & 0 & 9 
\end{tabular}
\end{center}


\end{enumerate}
\end{enumerate}
\begin{Verbatim}[numbers=left]
#Load the fgl data set in the MASS library. 

library(e1071)
data(fgl,package='MASS')
source('~/Dropbox/Tarleton/data_mining/class_notes/extras.R')
source('~/Dropbox/Tarleton/data_mining/generic_functions/dataset_ops.R')

# Split the data into 70\% training and 30\% testing data. 
# Predict the glass types in the test data using the following
# three methods, and compare the classification accuracies of 
# these methods.

splitset <- splitdata(fgl,0.7,F)
train <- splitset$train

#(a) Build an SVM and use the predict command.

a_model <- svm(type~., fgl[train,])
a_pred <- predict(a_model,fgl[-train,])
a_acc <- confmatrix(fgl$type[-train],a_pred)

#(b) Write a script implementing the one-against-rest approach 
#   with SVM's (much of the code you need is in the 'programming
#   one-against-rest' section of chapter5.R)

one_against_rest <- function(dset, v_idx, train, my_pred){
  tmpname <- names(dset)[v_idx]
  names(dset)[v_idx] <- 'type'
  levels_t <- levels(dset$type)
 
  tmpdset <- dset[train,]
  levels(tmpdset$type) <- c(levels_t, 'other')
  tmptype <- dset$type[train]
  preds <- list()
  for(idx in 1:length(levels_t) ){
  	name <- levels_t[idx]
  	
    index_v <- (tmpdset$type != name) * 1
    tmpdset$type[index_v == 1] <- 'other'    

 	model <- svm(type~., tmpdset)
 	tmppred <- predict(model, dset[-train,])
 	preds[[idx]] <- tmppred
 	tmpdset$type <- tmptype
    levels(tmpdset$type) <- c(levels_t,'other')
  }

  votes <- matrix(0, 
                 nrow = (nrow(dset[-train,])), 
                 ncol = length(levels_t) )

  for(idx in 1:length(levels_t)){
    name <- levels_t[idx]
    votes[,idx] <- (preds[[idx]] == name)
    for(i in 1:length(levels_t)){
      if(i != idx){
        votes[,idx] <- votes[,idx] + ( preds[[i]] == 'other' )
      }
    }
  }

  one_against_rest_pred <- levels_t[apply(votes, 1 ,which.max)]
  return(table(my_pred, one_against_rest_pred))
}

tab1 <- one_against_rest(fgl, ncol(fgl), train, fgl$type[-train])

#(c) Write a script implementing the one-against-one approach 
#   with SVM's.

comp_preds <- function(idx1, idx2, var, train, dset){
  tmpname <- names(dset)[var]
  names(dset)[var] <- 'my_var_name'
  myvar <- dset[,var]

  n1 <- levels(myvar)[idx1]
  n2 <- levels(myvar)[idx2]

  idx <- (((myvar == n1) | (myvar == n2) )) * 1
  idx[-train] = 0

  v1_v2_train <- dset[idx == 1,]
  v1_v2_model <- svm(my_var_name~., v1_v2_train)
  v1_v2_pred <- predict(v1_v2_model, dset[-train,], type='class')

  names(dset)[var] <- tmpname

  return(v1_v2_pred)
}

comp_votes <- function(preds, dset, v_idx){

  nr = length(preds[[1]][[1]])
  nc = length(levels(dset[,v_idx]))

  votes <- matrix(0,nrow=nr, ncol=nc)
  for(k in 1:nc){
    name1 <- levels(dset[,v_idx])[k]

    for(i in 1:nc){
      for(j in 1:nc){
        if(j != i){
          votes[,k] <- votes[,k] + (preds[[i]][[j]] == name1) * 1
        }
      }
    }
  }
  return(votes)
}

one_against_one <- function(v_idx, dset, train, predicted){
  tmpname <- (names(dset))[v_idx]
  varname <- 'mytempname'
  names(dset)[v_idx] <- 'mytempname'

  list_of_preds <- list()
  tmplist <- list()
  acc <- matrix(-1, nrow = ncol(dset), ncol = ncol(dset))
  len <- length( levels(dset$mytempname) )

  for(i in 1:len ){
    for(j in 1:len ){
      if(i != j){
        tmplist[[j]] <- comp_preds(i, j, v_idx, train, dset)
      }else{
      	tmplist[[j]] = rep(-1, nrow(dset) - length(train))
      }
    }
    list_of_preds[[i]] <- tmplist
  }
  votes <- comp_votes(list_of_preds, dset, v_idx)
  one_against_one_pred <- (levels(dset$mytempname))[apply(votes,1,which.max)]
  return(table(predicted, one_against_one_pred))
}

data(fgl,package='MASS')
tab2 <- one_against_one(ncol(fgl), fgl, train, fgl$type[-train])
\end{Verbatim}
\end{document}

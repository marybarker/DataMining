\documentclass[11pt]{article}
\usepackage{fullpage}
\usepackage{setspace}
\usepackage{amsmath}
\usepackage{fancyvrb}
\usepackage{enumerate}
\usepackage{listings}
\usepackage{pgfplots}
\usepackage{graphicx}
\usepackage{float}
\usepackage{multirow}
\usepackage[format=hang,labelsep=quad]{caption}
\usepackage{subfig}
\usepackage{array}
\usepackage{multirow}
\usepackage[final]{pdfpages}

\renewcommand\thesubfigure{\roman{subfigure}}


\begin{document}
\noindent\large{Math 5364}\\
\large{Data Mining 2}\\
\large{Homework 28}\\
\large{Mary Barker}

\begin{enumerate}
\item The data set cows.txt contains milk production 
	values for 300 (hypothetical) cows, 100 from 
	the Andrews farm, 100 from the Bailey farm, and 
	100 from the Carter farm. 
	
	\begin{enumerate}
	\item Import the data into SAS, and find the 
		average milk production, stratified by farm. 
		Also, obtain a histogram and qqplot of the 
		milk production values at each farm;

\begin{Verbatim}
data milkdata;
    infile '/folders/myshortcuts/sas_folder/cows.txt' dlm=',';
    input milk farm $;
    IF farm = 'Andrews' THEN Afarm = milk;
    IF farm = 'Bailey' THEN Bfarm = milk;
    IF farm = 'Carter' THEN Cfarm = milk;
proc means data=milkdata mean;
	var Afarm Bfarm Cfarm;
run;
\end{Verbatim}

	\item Perform and ANOVA to test whether the average 
		milk production at the three farms is the same;

\begin{Verbatim}
proc anova data=milkdata;
	class farm;
	model milk=farm;
run;
\end{Verbatim}


	\item Test whether the average milk production is 
		the same using PROC GLM;

\begin{Verbatim}
proc glm data=milkdata;
	class farm;
	model milk=farm;
run;
\end{Verbatim}
	\end{enumerate}

%$
\item Let $U(a, b)$ denote a uniform distribution on the 
	interval [a, b], and $N(\mu, \sigma^2)$ denote a normal 
	distribution with mean $\mu$ and variance $\sigma^2$. Let 
	$X_{i1} \mathtt{\sim} U(0, 100)$, $X_{i2} \mathtt{\sim} U(30, 70)$, and 
	$\epsilon_i \mathtt{\sim} N(0, 1) for i = 1, ... , 1000$. Also, suppose 
	$X_{i3}$ is a categorical variable taking the values ``A'', 
	``B'', and ``C'' with probabilities 0.5, 0.35, and 0.15 
	respectively. Finally, assume that all of the random 
	variables $X_{ij}$ and $\epsilon_i$ are statistically independent 
	and define 

	\begin{equation*}
	Y_i = 150 + 8 X_{i1} + 6 X_{i2} + 0.25 X_{i2}^2 - 7 X_{i1} X_{i2}
			  + 5I(X_{i3} = "B") + 10I(X_{i3} = "C") + \epsilon_i
	\end{equation*}
  
	Recall that $I$ is the indeicator function. 
	e.g., $I(X_{i3} = 'B') = 1$ if $X_{i3} = 'B'$, and 0 otherwise;

	\begin{enumerate}
	\item Use SAS to simulate values of all random variables 
		described above;

\begin{Verbatim}
data simdata;
call streaminit(123);
do i = 1 to 1000;
	x1 = rand('UNIFORM');
	x2 = 30 + 70 * rand('UNIFORM');
	u = rand('UNIFORM');
	if u <= 0.5 then do;
		x3 = 'A';
		I_b = 0;
		I_c = 0;
	end;
	else if u <= 0.85 then do;
		x3 = 'B';
		I_b = 1;
		I_c = 0;
	end;
	else do;
		x3 = 'C';
		I_b = 0;
		I_c = 1;
	end;
	eps = rand('NORMAL');
	y = 150. + 8 * x1 + 6 * x2 + 0.25 * x2**2 - 7 * x1 * x2 + 
			5 * I_b + 10 * I_c + eps;
	output;
end;
run;
\end{Verbatim}

	\item Verify that $X_{i1}$, $X_{i2}$, and $e_i$ have the 
		distributions given above by plotting histograms 
		for these variables;

\begin{Verbatim}
proc univariate data=simdata;
	histogram x1;
	histogram x2;
	histogram eps/normal;
run;
\end{Verbatim}
		
	\item Verify that the observed frequencies of the different 
		levels of $X_{i3}$ are approximately equal to those stated 
		in the problem;

\begin{Verbatim}
proc freq data=simdata;
	tables x3;
run;
\end{Verbatim}

	\item Fit the given regression equation to your simulated 
		data, and verify that the estimated coefficients agree 
		with those stated in the problem; 


\begin{Verbatim}
proc glmselect data=simdata;
	class x3;
	model y = x1 x2 x2*x2 x1*x2 x3;
run;
\end{Verbatim}

\end{enumerate}
\end{enumerate}

\includepdf[pages={1-}]{hw28_output.pdf}
\end{document}

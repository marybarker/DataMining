\documentclass[11pt]{article}
\usepackage{fullpage}
\usepackage{setspace}
\usepackage{amsmath}
\usepackage{fancyvrb}
\usepackage{enumerate}
\usepackage{pgfplots}
\usepackage{graphicx}
\usepackage{float}
\usepackage{multirow}

\begin{document}
\noindent\large{Math 5365}\\
\large{Data Mining 1}\\
\large{Homework 6}\\
\large{Mary Barker}
\newline
\begin{enumerate}
\item 

\begin{enumerate}
\item This problem will apply knn to the wdbc.data data set
 Standardize the data and verify that the column means 
 are equal to zero and the column standard deviations 
 are equal to one afterwards.\\

The code for this and all of the following problems is attached at the 
end of the document. 

For this case, each column was successfully scaled such that the mean 
value of all of the values in each column is 0 and the standard deviation 1. 

\item  Split the data into 70\% training and 30\% test data.\\

As in homework 5, the data was successfully split into training and testing data. 

\item Calculate the test error rate for predicting breast 
 cancer diagnosis using knn with k = 3, and find a 95\% 
 confidence interval for this error rate.\\

The error when using knn with k = 3 is 0.04093567. 
The exact confidence interval for this error rate is (0.9399628, 0.9742064).

\item Compare the test error rates of  knn with k = 3 and 
 rpart, and determine if there is a statistically 
 significant difference between them. \\

The error when using rpart is 0.04678363, with an exact confidence interval 
of (0.9317086, 0.9684996).

Using an exact McNemar test on the two prediction models resulted in a p-value 
of roughly 0.003 which is below the threshold $\alpha = 0.05$. Thus there is 
a statistically significant difference between the two. 
\end{enumerate}

\item 
\begin{enumerate}

\item Use knn.cv to estimate the error when k = 3. \\

The error was computed as 0.03514938 and therefore the error rate is 
approximately 3.5\%.

\item Use knn.cv to find the value of k = 1, 2, ... , 10 that 
 minimizes the error rate. Let the optimal value be k$_0$.\\

The optimal value k$_0$ is 4.

\item Estimate the error rate of knn with k = k$_0$ using 10-fold
 cross-validation.\\

The average error rate using 10-fold cross-validation is 0.03342732
and therefore the average error rate is approximately 3.3\%.

\item Estimate the error rate of knn with k = k$_0$ using the 
 bootstrap with b = 100.\\

The average error is 0.02666667, or 2.$\bar6$\%.

\end{enumerate}

\item Bonus: Write your own function for performing k-nearest 
 neighbors classification and compare the results you obtain 
 with knn. Here are some guidelines. 
\begin{enumerate}

\item To keep things simple, you can assume there are only two 
 class labels and k is odd, so you don't have to worry about 
 ties. On the other hand, breaking ties isn't too hard. It 
 might be interesting to figure out how to do it. (hint:runif).

\item A good starting point would be to write a function called 
 distancematrix, whcih accepts matrices x$_\text{train}$ and x$_\text{test}$. 
 It returns a matrix D such that D$_{ij}$ is the euclidean 
 distance between the $i$th row of x$_\text{train}$ and the $j$th row of 
 x$_\text{test}$. Once you have the distance matrix D, finding nearest
 neighbors and so on should be pretty easy.

\end{enumerate}

Using my own nearest neighbors algorithm with k = 3 on multiple runs with 
as many subdivisions of the data, and also over a few hand-manufactured 
test cases for hands-on validation resulted in the exact same error as with 
the knn function every time, and the correct distribution for each of the 
hand-computable test cases. The exact values were to be expected, since the 
value of k used each time was odd, so there were no ties to break. 
However, in case of potential ties, there is logic in place to arbitrate the 
outcome in a nonbiased way. 
In case of a tie the program chooses the value of the class to choose 
based on the value of a randomly generated number. 

\end{enumerate}
\begin{Verbatim}
# Data Mining hw 6
library(class)
library(stats)
library(exact2x2)
library(exactci)
library(rpart)

wdbc <- read.table("~/Dropbox/Tarleton/data_mining/hw05/wdbc.data",
header=FALSE,sep=",")
#1 This problem will apply knn to the wdbc.data data set
# a) Standardize the data and verify that the column means 
#    are equal to zero and the column standard deviations 
#    are equal to one afterwards.
wdbc <-wdbc[ ,-c(1) ]
n = nrow(wdbc)
x = seq(from=2,to=ncol(wdbc),by=1)

sdized <- standardize(wdbc, x)
apply(sdized[,x],2,sd)
apply(sdized[,x],2,mean)


# b) Split the data into 70% training and 30% test data.
splitset <- splitdata(sdized, 0.7, FALSE)
training <- splitset$traindata
testing <- splitset$testdata
train = splitset$train

# c) Calculate the test error rate for predicting breast 
#    cancer diagnosis using knn with k = 3, and find a 95% 
#    confidence interval for this error rate.
pred_knn = knn(train=sdized[train,x],test=sdized[-train,x],
cl=sdized$V2[train], k=3)

M_list = confmatrix(wdbc$V2[-train], pred_knn)

error_knn = M_list$error

exact_knn = binom.test(x=n - round(error_knn*n, digits=0), n=n, p=0.05)

# d) Compare the test error rates of  knn with k = 3 and 
#    rpart, and determine if there is a statistically 
#    significant difference between them. 

tree_rpart = rpart(V2~., sdized[train,])
pred_rpart = predict(tree_rpart, newdata=sdized[-train,],type="class")
M_list = confmatrix(sdized$V2[-train], pred_rpart)
error_rpart = M_list$error

exact_rpart = binom.test(x=n-round(error_rpart*n,digits=0),n=n,p=0.05)

accv1 <- (sdized[-trainv,]$V2 == predict(tree_rpart, sdized[-trainv,], type='class'))
accv2 <- (sdized[-trainv,]$V2 == pred_knn)
mcnemartable = table(accv1, accv2)
mcnemar.exact(mcnemartable)


#2
# a) Use knn.cv to estimate the error when k = 3
V2 = sdized$V2
pred_knn_2 = knn.cv(train = sdized[,x], cl = V2, k = 3)
M_list = confmatrix(V2, pred_knn_2)
error_knn_2 = M_list$error

exact_knn_2 = binom.test(x=n-round(error_rpart*n,digits=0),n=n,p=0.05)

print(c(error_knn, error_rpart, error_knn_2))
print(c(exact_knn, exact_rpart, exact_knn_2))

# b) Use knn.cv to find the value of k = 1, 2, ... , 10 that 
#    minimizes the error rate. Let the optimal value be k_0
accvect = rep(0,10)
for(k in 1:10){
  pred_knn_2 = knn.cv(train = sdized[,x], cl = V2, k = k)
  accvect[k] = confmatrix(V2, pred_knn_2)$accuracy
}
k_0 = which.max(accvect)

# c) Estimate the error rate of knn with k = k_0 using 10-fold
#    cross-validation.
k = 10
size = ceiling(n/k)
folds = sample(rep(1:k,size))
myvec = folds[1:n]
tmperror10 = 0
for(i in 1:10){
  trn <- sdized[myvec!=i,]
  tst <- sdized[myvec==i,]
  tmpknn = knn(trn[,x],tst[,x],V2[myvec!=i],k_0)

  M_list = confmatrix(V2[myvec==i], tmpknn)

  tmperror10 = tmperror10 + M_list$error
}
tmperror10 = tmperror10 / 10
print(tmperror10)

# d) Estimate the error rate of knn with k = k_0 using the 
#    bootstrap with b = 100.
k = 100
size = ceiling(n/k)
folds = sample(rep(1:k,size))
myvec = folds[1:n]
tmperror100 = 0
for(i in 1:100){
  trn <- sdized[myvec!=i,]
  tst <- sdized[myvec==i,]
  tmpknn = knn(trn[,x],tst[,x],V2[myvec!=i],k_0)

  M_list = confmatrix(V2[myvec==i], tmpknn)

  tmperror100 = tmperror100 + M_list$error
}
tmperror100 = tmperror100 / 100
print(tmperror100)

# 3. Bonus

distancematrix = function(x_train, x_test){
  if(ncol(x_train) != ncol(x_test)){
    print("error in function distancematrix: incompatible matrix sizes")
  }

  d_mat <- matrix(,nrow=nrow(x_train),ncol=nrow(x_test))

  for(i in 1:nrow(x_train)){
    for(j in 1:nrow(x_test)){
      x1 = as.numeric(x_train[i,])
      x2 = as.numeric(x_test[j,])
      d = sqrt( t(x1 - x2) %*% (x1 - x2))
      d_mat[i,j] = d
    }
  }
  return(d_mat)
}

# quick hand-verifiable test for distancematrix
# a <- matrix( seq(1:6),3)
# b <- matrix( seq(from=2,to=16,by=2),4)
# d <- distancematrix(a, b)


find_closest_nbrs = function(dset, trainidx, vidx, k){
  if(k > length(trainidx)){
    print("Error in function find_closest_nbrs:k too big")
    print("... resetting k = size of training data")
    k = length(trainidx)
  }
  train <- dset[trainidx,-c(vidx)]
  test <- dset[-trainidx,-c(vidx)]

  D <- distancematrix(train, test)
  closest_vals <- matrix(,nrow(test),k)
  testvals <- rep('+', nrow(test))

  for(i in 1:nrow(test)){
    # get row numbers for the k traindata rows closest to each testdata column i
     mylist <- sort(D[,i])[1:k]
     nidx <- vector()
 
     for(j in 1:k){
       nidx <- c(nidx, which(D[,i] == mylist[j]) )
     }
 
     # then find predominating class value for that set of closest rows
     myvals <- dset[trainidx[nidx],vidx]
     t1 <- levels(dset[,vidx])[1]
     t2 <- levels(dset[,vidx])[2]
     n1 <- sum( (myvals == t1) * 1)
     n2 <- length(myvals) - n1
 
     if(n1 > n2){
       testvals[i] <- t1
     }else if(n2 > n1){
       testvals[i] <- t2
     } else {
       myrand <- runif(1)
       if(myrand > 0.5){
         testvals[i] <- t1
       }else{
       	testvals[i] <- t2
       }
     }
   }
   print(testvals)
   mytable <- confmatrix(dset[-trainidx, vidx], testvals)
   return(mytable)
}
t <- find_closest_nbrs(wdbc, trainv, 1, 3)
tknn <- knn(wdbc[trainv,-c(1)],wdbc[-trainv,-c(1)], wdbc$V2[trainv], 3)
confmatrix(wdbc$V2[-trainv],tknn)$error
t$error

# test case that is hand-verifiable for closest_nbrs routine
M <- c("M", "M","B","M","B","B","B","M")
V2 <- c(20, 3, 4, 19, 38, 6, 15, 47)
V3 <- c(6, 8, 1, 5, 2, 3, 0, 9)
V4 <- c(150, 200, 108, 215, 198, 170, 101, 155)
d1 <- data.frame(M, V2, V3, V4)

M <- c("M", "M","B","M")
V2 <- c(10, 5, 26, 2)
V3 <- c(4, 1, 6, 5)
V4 <- c(190, 201, 105, 177)
d2 <- data.frame(M, V2, V3, V4)

total <- rbind(d1, d2)
tt1 <- 1:8
val = 1

t_test <- find_closest_nbrs(total, tt1, 1, 3)

x1 = 2:4
t_pred_knn = knn(train=total[tt1,x1],test=total[-tt1,x1],cl=total$M[tt1], k=3)
t_test_knn <-confmatrix(total[-tt1, 1], t_pred_knn)

####################################################
# test case that is hand-verifiable for closest_nbrs routine
M <- c("M", "B","B","M","B","B","B","M","M", "B","B","M","B","B","B","M")
V2 <- c(20, 3, 4, 3.3, 38, 6, 15, 47,20, 3, 4, 3.3, 38, 6, 15, 47)
V3 <- c(6, 8, 1, 7.5, 2, 3, 0, 9,6, 8, 1, 7.5, 2, 3, 0, 9)
V4 <- c(150, 200, 108, 201, 198, 170, 101, 155,150, 200, 108, 201, 198, 170, 101, 155)
d1 <- data.frame(M, V2, V3, V4)

M <- c("M", "M","B","M","M", "M","B","M")
V2 <- c(10, 5, 26, 2, 11, 15, 46, 12)
V3 <- c(4, 1, 6, 5, 4, 1, 6, 5)
V4 <- c(190, 201, 105, 177, 186, 197, 101, 174)
d2 <- data.frame(M, V2, V3, V4)

total <- rbind(d1, d2)
tt2 <- 1:8
val = 1

t_test <- find_closest_nbrs(total, tt2, 1, 3)
\end{Verbatim}

\end{document}
